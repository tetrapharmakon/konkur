	Ai sensi del bando di selezione \myTextField[.5cm]{} \myTextField[.25cm]{sottoscritt}
	dichiara sotto la propria responsabilità che quanto di seguito affermato
	corrisponde a verità:
	\begin{itemize}
		\item di essere in possesso della seguente \myTextField{cittadinanza};
		\item di essere iscritto/a nelle liste elettorali del comune \myTextField{di}
		      (\emph{solo per cittadini italiani});\footnote{Indicare eventualmente i motivi
			      della mancata iscrizione.}

		      \myTextField[.9\textwidth]{}
		\item di godere dei diritti civili e politici nello Stato di appartenenza o di
		      provenienza;
		\item di non aver riportato condanne penali;\footnote{Dichiarare le eventuali
			      condanne penali riportate.}

		      \myTextField[.9\textwidth]{}
		\item di trovarsi nella seguente condizione nei riguardi degli obblighi
		      \myTextField{militari:};
		\item di non essere stato/a destituito/a o dispensato/a dall'impiego presso una
		      pubblica amministrazione, ovvero di non essere stato dichiarato decaduto da un
		      impiego statale, ai sensi dell'art. 127, primo comma, lettera d) del decreto
		      del Presidente della Repubblica 10.01.1957, n. 3;
		\item di eleggere domicilio, agli effetti della selezione, in
		      \begin{center}
			      \titleTextField[6cm]{}
		      \end{center}
		      \myTextField[5cm]{e-mail:}, \myTextField[5cm]{PEC:}, \myTextField{telefono},
		      riservandosi di comunicare tempestivamente ogni eventuale variazione dello stesso;
		\item di non essere professore o ricercatore universitario di ruolo, ancorché
		      cessato dal servizio;
		\item di non avere un grado di parentela o di affinità fino al quarto grado
		      compreso, o un rapporto di coniugio, con un professore o ricercatore appartenente
		      al dipartimento o alla struttura che richiede la procedura di selezione pubblica,
		      ovvero con il Rettore, il Direttore Generale o con un componente del Consiglio
		      di Amministrazione dell’Ateneo;
		\item di avere adeguata conoscenza della lingua italiana (\emph{solo per i
			      cittadini stranieri}).
	\end{itemize}
